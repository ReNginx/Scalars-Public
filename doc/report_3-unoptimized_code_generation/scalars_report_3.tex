\section{Design}

The unoptimized code generation component of the the Scalars Decaf Compiler is built via the following procedure:
\begin{enumerate}
    \item Convert high-level IR tree into low-level IR by flattening expression trees.
    \item Generate a control-flow graph based on the low level IR tree.
    \item Generate |x86-64| assembly from the CFG.
\end{enumerate}

\subsection{Expression Tree Flattening}
For each maximal |Expression| tree |t| rooted at an |Operation|, we need to flatten |t| such that |t| can be represented by a series of 3-address operations involving temporary variables. The flattening process is described by the following:

For each |Operation| |t|, we augment |t| with two fields, |.eval| and |.block|, where
\begin{enumerate}
    \item |.eval| is a |Location| instance that tracks the temporary variable containing the result of the |Operation|, and
    \item |.block| is a |Block| instance that contains a list of temporary variable declarations in |.declarations| field, and a list of 3-address operations for the |Expression| tree rooted at |t| in |.statements| field.
\end{enumerate}

When we process each |Operation| |t|, |.eval| is initialized as an empty location, and|.block| is initialized as an empty block. We then recursively apply flattening to the children of |t| (i.e. |child| for unary operators, |lhs| and |rhs| for binary operators,and |condition|, |ifTrue|, |ifFalse| for the ternary operator). We then populate |t.block| by combining the |.block| fields of |t|'s children. We do so for both the |.declarations| and the |.statements| fields of the block.

After populating |t.block|, we declare a new temporary variable |tmp| and add its declaration to |t.block.declarations|. We set |t.eval| as a |Location| instance of |tmp|. Finally, we add an assignment operation to |t.block.statements| for calculating |tmp| from the |.eval| fields of |t|'s children.

Throughout the recursive flattening process, we keep a global iterator that increments whenever we initialize a new temporary variable. We use this technique to generate a unique name for each temporary variable in the program.

\subsection{Control-Flow Graph Generation}

A control-flow graph is generated from the low-level IR. The processes is separated into 3 stages.
\begin{enumerate}
    \item Short-circuiting logical expressions.
    \item Destructuring logical constructs.
    \item Joining adjacent blocks.
\end{enumerate}

\subsubsection{Short-Circuiting}

Given a logical expression |t| that has a depth $d \geq 2$, the short-circuiting of |t| is done as follows:
\begin{enumerate}
    \item Examine the type (i.e. unary, binary or ternary) and the operator (i.e. |&&|, \lstinline{||}, etc.) of the root of |t|. For each operator, generate a CFG snippet from a pre-specified template.
    \item Recurse into each conditional in the template and repeat the same operation.
    \item If the expression for a conditional is atomic (i.e. not a compound logical expression), replace the conditional in the template with the actual expression.
    \item Once all conditionals are filled in a CFG snippet |s|, add |s| to its parent snippet by replacing |s|'s corresponding conditional in the parent snippet with |s|. Connect the inbound/outbound edges of |s| with its parent's edges as necessary.
\end{enumerate}

\subsubsection{Destructuring}

Logical constructs are destructured as outlined in the lecture notes.

\subsubsection{Joining Adjacent Blocks}

We scan for adjacent blocks of declarations and 3-address statements, and join them into a single node in the CFG. After this process, every node in the CFG is represented by one of the following 6 categories:
\begin{enumerate}
    \item |VirtualCFG|: Start and end nodes that do not contain statements.
    \item |CFGBlock|: Basic block in CFG that does not contain conditional statements.
    \item |CFGConditional|: Basic block in CFG that represents a single conditional statement.
    \item |CFGMethod|: Basic block in CFG that represents a method declaration.
    \item |CFGMethodCall|: Basic block in CFG that represents a method call.
    \item |CFGProgram|: Basic block in CFG that represents a program.
\end{enumerate}

\subsection{\lstinline{x86-64} Assembly Generation}

The CFG is linearized, and from which |x86-64| assembly is generated. When generating code for method calls, the Linux/GCC |x86-64| calling convention is used.

\newpage

\section{Extras}

\subsection{Producing Debug Information}

When using the |--target=assembly| target, the user may use the |--debug| switch to pretty-print the low-level intermediate representation tree. For each node of the IR, its corresponding IR class, row/column and other relevant info are displayed. Additionally, for each expression tree, a list of flattened code, as well as the declarations of any associated temporary variables are displayed.

\subsection{Build/Run Scripts}

We re-implemented the build and run scripts. The scripts and their functionalities are specified as follows:

\begin{enumerate}
    \item |build.sh|: Attaches |scala| if the current machine's FQDN ends with |.mit.edu|. Invokes |setenv.sh|. Passes all additional parameters to |ant|.
    \item |run.sh|: Attaches |scala| if the current machine's FQDN ends with |.mit.edu|. Invokes |setenv.sh|. Sets |JAVA_OPTS| appropriately. Invokes program and passes all additional parameters to program.
    \item |setenv.sh|: Checks if the current machine's FQDN ends with |.mit.edu|. If so, populate the current shell with the environment variables specified in |athena.environment|; otherwise, populate the current shell with the environment variables specified in |local.environment|.
\end{enumerate}

Each |.environment| file follows a one-line-per-variable format. Variable expansions (e.g. |$var| are allowed, as long as the variable is defined in the current shell or in previous lines of the environment file.

The |athena.environment| is attached as follows:
    \begin{lstlisting}[language=Bash]
SCALA_HOME=/mit/scala/scala/scala-2.11.2/
    \end{lstlisting}

\section{Difficulties}
\begin{enumerate}
    \item It is challenging to design and implement an effective expression tree flattening scheme. An important aspect of expression tree flattening is to maintain information necessary in generating the CFG, especially for short-circuiting and destructuring. In our final solution, each node |t| of an expression tree contains all 3-address code and temporary variable declarations needed to compute the subtree rooted at |t|. This design choice borrows from the idea of persistent data structures, and it allows us to augment the IR with flattened code, rather than replacing it. Hence we are able to retain the original expression tree structure for short-circuiting and destructuring.
    \item Performing short-circuiting in CFG generation.
\end{enumerate}

\newpage

\section{Contribution}

\subsection{Jack}
\begin{enumerate}
    \item Designed and implemented control-flow graph generation.
\end{enumerate}

\subsection{Allen}
\begin{enumerate}
    \item Designed and implemented expression tree flattening.
    \item Added command line interface for |--target=assembly| switch and its |--debug| variant. Added pretty-printing. Created and maintained run and build scripts.
    \item Completed documentation.
\end{enumerate}

\subsection{Hanxiang}
\begin{enumerate}
    \item Designed and implemented |x86-64| assembly generation.
\end{enumerate}